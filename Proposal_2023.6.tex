\documentclass[10pt]{beamer}
\usetheme{CambridgeUS}
\usecolortheme{beaver}
\usepackage{graphicx}
\usepackage{color}
\usepackage{mathptmx}
\usepackage{geometry}
\usepackage{multicol}
\usepackage{hyperref}

\colorlet{shadecolor}{blue!15}
\newtheorem{prop}{Proposition}
\newenvironment{propc}[1]
{\begin{shaded}\begin{prop}}
		{\end{prop}\end{shaded}}

\title[Global Monetary Policy Spillover Through Trade Channel]{Global Monetary Policy Spillover Through Trade Channel}
\author[Yao\&Lu (2023)]{\large Lingfei LU, Jingbo YAO \\ \vspace{0.5cm} Department of Economics, HKUST \\ \vspace{0.5cm} HKUST-Jinan Macro Student Symposium 2023}
\date{\today}
\logo{\includegraphics[scale=0.15]{UST4C_L3.eps}}

\begin{document}
	
    \begin{frame}[plain]
	\maketitle {}
    \end{frame}

%----------------------------------------

\section{Introduction}

\begin{frame}{Introduction}
	\begin{itemize}
		\item 
	\end{itemize}
\end{frame}

\section{Model}

\section{Empirical}

\begin{frame}{Data: Customs Transaction Records}
	\begin{itemize}
		\item The first dataset is the transaction level records from the General Administration of Customs of China (GACC) during 2000-2011.
		\begin{itemize}
			\item This dataset provides information on
			all Chinese trade transactions, including import and export values, quantities, units, product names and codes, source and destination countries, and types of enterprises.
			\item The categories of products are coded by the Harmonized Coding and Description System (HS) from the World Customs Organization (WCO).
		\end{itemize}
		\item We drop unwanted observations referring to \cite{lmx2015}:
		\begin{itemize}
			\item products with inconsistent missing information of unit or quantity
			\item special product categories such as arms (HS2=93), antiques (HS2=97), and special categories (HS2=98 and 99)
			\item transactions existing for only one year without any change over time.
		\end{itemize}
	\end{itemize}
\end{frame}

\begin{frame}{Data: Firm-level Information}
	\begin{itemize}
		\item The second dataset is the Annual Survey of Chinese Industrial Enterprises (CIE) database from the National Bureau of Statistics of China (NBSC).
		\begin{itemize}
			\item This dataset covers all state-owned enterprises and above-scale firms
			with annual sales of more than 5 million RMB from 1999 to 2007.
			\item The data provide details about firms’ identification code, ownership, industry
			type, and about 80 other variables in the balance sheet, including the number of employees, total wage payments, the value of fixed assets, sales income, total operation inputs, etc.
		\end{itemize}
		\item To merge it with customs records, we match the identification codes based on the contact information of firms as in \cite{fan-li-yeaple2015}.
	\end{itemize}
\end{frame}

\begin{frame}{Data: Monetary Policy Surprise}

\end{frame}

\end{document}