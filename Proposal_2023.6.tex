\documentclass[10pt]{beamer}
\usetheme{CambridgeUS}
\usecolortheme{beaver}
\usepackage{graphicx}
\usepackage{color}
\usepackage{mathptmx}
\usepackage{geometry}
\usepackage{multicol}
\usepackage{hyperref}

\colorlet{shadecolor}{blue!15}
\newtheorem{prop}{Proposition}
\newenvironment{propc}[1]
{\begin{shaded}\begin{prop}}
		{\end{prop}\end{shaded}}

\title[Monetary Policy Spillover Through Trade Channel]{Global Monetary Policy Spillover Through Trade Channel}
\author[Yao\&Lu (2023)]{\large Lingfei LU, Jingbo YAO \\ \vspace{0.5cm} Department of Economics, HKUST \\ \vspace{0.5cm} HKUST-Jinan Macro Student Symposium 2023}
\date{\today}
\logo{\includegraphics[scale=0.15]{UST4C_L3.eps}}

\begin{document}
	
    \begin{frame}[plain]
	\maketitle {}
    \end{frame}

%----------------------------------------

\section{Introduction}

\begin{frame}{Introduction}
	\begin{itemize}
		\item 
	\end{itemize}
\end{frame}

\section{Model}

\section{Empirical}

\begin{frame}{Data: Monetary Policy Surprise}
    \begin{itemize}
        \item First, we use a U.S. monetary policy shock series developed by \cite{brw2021}
        that stably bridges periods of conventional and unconventional policymaking, is largely unpredictable, and contains no significant central bank information effect
    \end{itemize}
\end{frame}

\begin{frame}{Data: Customs Transaction Records}
	\begin{itemize}
		\item Second, we use the transaction level records from the General Administration of Customs of China (GACC) during 2000-2011.
		\begin{itemize}
			\item This dataset provides information on
			all Chinese trade transactions, including import and export values, quantities, units, product names and codes, source and destination countries, and types of enterprises.
			\item The categories of products are coded by the Harmonized Coding and Description System (HS) from the World Customs Organization (WCO).
		\end{itemize}
		\item We drop unwanted observations referring to \cite{lmx2015}:
		\begin{itemize}
			\item products with inconsistent missing information of unit or quantity
			\item special product categories such as arms (HS2=93), antiques (HS2=97), and special categories (HS2=98 and 99)
			\item transactions existing for only one year without any change over time.
		\end{itemize}
	\end{itemize}
\end{frame}

\begin{frame}{Data: Firm-level Information}
	\begin{itemize}
		\item Third, we use annual surveys of Chinese manufacturing firms conducted by the National Bureau of Statistics of China (NBSC).
		\begin{itemize}
			\item This dataset covers all state-owned enterprises and above-scale firms with annual sales of more than 5 million RMB from 1999 to 2007.
			\item The data provide details about firms’ identification code, ownership, industry type, and 80 other balance sheet variables, including the number of employees, total wage payments, fixed assets, sales income, total operation inputs, etc.
		\end{itemize}
		\item To merge it with customs records, we match the identification codes based on the contact information of firms as in \cite{fan-li-yeaple2015}.
	\end{itemize}
\end{frame}

\begin{frame}{}
    
\end{frame}


%%%%%%%%%%%%%%%%%%%%%%%%%%%%%%%%%%%%%%%%%%%%%%%%%%
\section{Model}

\begin{frame}{Conceptual framework}
    
\end{frame}

\begin{frame}{Households}
\fontsize{8}{8}\selectfont

\begin{itemize}
    \item A consumer receives utility from consumption, $C_{t}$, and disutility from labor, $L_{t}$. Its lifetime utility is
\end{itemize}


$$
V_{t}=\max \mathbb{E}_{t} \sum_{j=0}^{\infty} \beta^{j}\left[\frac{C_{t+j}{ }^{1-\sigma}-1}{1-\sigma}-\psi \frac{L_{t+j}^{1+\chi}}{1+\chi}\right]
$$

\begin{itemize}
    \item where $\beta \in(0,1)$ is the discount factor, $\sigma>0$ is the inverse elasticity of intertemporal substitution, $\chi \geq 0$ is the inverse Frisch elasticity, and $\psi>0$ is a scaling parameter. The saver's budget constraint is:
\end{itemize}


$$
P_{t} C_{t}+S_{t}+\mathbb{E}_{t} \Lambda_{t, t+1} \mathcal{D}_{t+1} \leq W_{t} L_{t}+R_{t-1} S_{t-1}+\mathcal{D}_{t}+P_{t} D_{t}+P_{t} T_{t}
$$



\begin{itemize}
    \item The consumer earns nominal income from labor, with a wage of $W_{t}$, receives dividends from ownership in firms, $D_{t}$, and receives a lump-sum transfer from the fiscal authority, $T_{t}$. It can save via one-period nominal bonds, $S_{t}$, that pay gross nominal interest rate, $R_{t}$. The consumption index is a composition of home and foreign goods, so
\end{itemize}



$$
C_{t}\equiv\left(C_{H, t}\right)^{1-\gamma}\left(C_{F, t}\right)^{\gamma}
$$

\begin{itemize}
    \item and the corresponding CPI $P_{t}$ can be expressed as
\end{itemize}

$$
P_{t}=k^{-1} P_{H, t}^{1-\gamma} P_{F, t}^{\gamma}, \ where \ k \equiv(1-\gamma)^{(1-\gamma)} \gamma^{\gamma}
$$

\end{frame}

\begin{frame}{Households (continued)}
\fontsize{8}{8}\selectfont

\begin{itemize}
    \item Since the prices are set in the producer currency, law of one price holds. We will have $P_{F, t}=P_{F, t}^{*} \mathcal{E}_{t}$ and $P_{H, t}^{*}=\frac{P_{H, t}}{\mathcal{E}_{t}}$, where $\mathcal{E}_{t}$ is the nominal exchange rate, defined as the price of foreign currency in terms of domestic currency. 
\end{itemize}


$$
\mathcal{E}_{t}=\frac{P_{t}}{P_{t}^{*}}
$$



\begin{itemize}
    \item The terms of trade is then given by
\end{itemize}


$$
\mathcal{T}_{t} \equiv \frac{P_{F, t}}{P_{H, t}}=\frac{P_{F, t}^{*} \mathcal{E}_{t}}{P_{H, t}}
$$

\begin{itemize}
    \item The first-order conditions for consumption allocation, labor supply and intertemporal optimization are
\end{itemize}



$$
\begin{aligned}
& P_{H, t} C_{H, t}=(1-\gamma) P_{t} C_{t} \\
& P_{F, t} C_{F, t}=\gamma P_{t} C_{t} \\
& \beta\left(C_{t} / C_{t-1}\right)^{-\sigma}\left(P_{t-1} / P_{t}\right)=\Lambda_{t-1, t} \\
& 1 / R_{t}=\mathbb{E}_{t} \Lambda_{t, t+1} \\
& \frac{W_{t}}{P_{t}\left(C_{t}\right)^{\sigma}}=\psi L_{t}^{\chi}
\end{aligned}
$$
\end{frame}

\begin{frame}{Producer-final goods}

\begin{itemize}
    \item Each final goods firm in the home country uses a continuum of intermediate goods to produce output, according to the following CES technology:
\end{itemize}


$$
Y_{t}=\left(\int_{0}^{1} Y_{t}(f)^{(\epsilon-1) / \epsilon} \mathrm{d} f\right)^{\epsilon /(\epsilon-1)}
$$

\begin{itemize}
    \item where $Y_{t}$ denotes aggregate output, while $Y_{t}(f)$ is the input produced by intermediate goods firm $f$. Both variables are normalized by population size $1-\gamma$, i.e., they are expressed in per capita terms. Profit maximization, taking the price of the final good $P_{\mathrm{H}, t}$ as given, implies the set of demand equations:
\end{itemize}

$$
Y_{t}(f)=\left(\frac{P_{\mathrm{H}, t}(f)}{P_{\mathrm{H}, t}}\right)^{-\epsilon} Y_{t}
$$

\begin{itemize}
    \item as well as the domestic price index
\end{itemize}


$$
P_{\mathrm{H}, t}=\left(\int_{0}^{1} P_{\mathrm{H}, t}(f)^{1-\epsilon} \mathrm{d} f\right)^{1 /(1-\epsilon)}
$$

\end{frame}

\begin{frame}{Producer-intermediate goods}

\fontsize{8}{8}\selectfont

\begin{itemize}
    \item Each intermediate firm $f$ in the home economy has the following production technology.
\end{itemize}

$$
Y_{t}(f)=A_{t} L_{t}(f)
$$

\begin{itemize}
    \item where $A_{t}$ is an exogenous country-specific productivity disturbance obeying a known stochastic process. The firm must borrow an amount $W_tL_t(f)$ from intermediaries at the gross nominal interest rate $R_t$, so the nominal cost of labor is $R_tW_t$. The real marginal cost is the same for all firm. This implies the minimized nominal marginal cost of production is
\end{itemize}

$$
M C_{t}(f)=\frac{w_{t} P_{t}R_{t}}{A_{t}}
$$

\begin{itemize}
    \item where $w_{t}=\frac{W_{t}}{P_{t}}$ is the real wage.
\end{itemize}


\begin{itemize}
    \item Export prices are assumed to be set in producer currency (PCP). Intermediate firms are subject to a Calvo (1983) pricing friction in each period, firm $f$ resets its prices with probability $1-\phi$, with $\phi \in[0,1]$.
\end{itemize}

$$
\max _{\tilde{P}_{H, t}(f)} \mathbb{E}_{t} \sum_{j=0}^{\infty} \phi^{j} \Lambda_{t, t+j}^{s}\left[\left(1-\tau_{H}\right)\left(\frac{\tilde{P}_{H, t}(f)}{P_{H, t+j}}\right)^{-\epsilon} Y_{t+j} \tilde{P}_{H, t}(f)-\left(\frac{\tilde{P}_{H, t}(f)}{P_{H, t+j}}\right)^{-\epsilon} Y_{t+j} M C_{t+j}(f)\right]
$$

\begin{itemize}
    \item where $\tau_{H}=\frac{1}{\phi-1}$ is a subsidy imposed by the government to eliminate the steady state monopolistic distortion.
\end{itemize}

\end{frame}

\begin{frame}{Producer-intermediate goods (continued)}

\begin{itemize}
    \item With the symmetric assumption, $\tilde{P}_{H, t}(f)=\tilde{P}_{H, t}$. The optimal reset price satisfies:
\end{itemize}


$$
\begin{aligned}
\tilde{P}_{H, t} & =\frac{\epsilon}{\epsilon-1} \frac{X_{1, t}}{X_{2, t}} \\
X_{1, t} & =P_{H, t}^{\epsilon} M C_{t} Y_{t}+\phi \mathbb{E}_{t} \Lambda_{t, t+1} X_{1, t+1} \\
X_{2, t} & =P_{H, t}^{\epsilon} Y_{t}+\phi \mathbb{E}_{t} \Lambda_{t, t+1} X_{2, t+1}
\end{aligned}
$$

\begin{itemize}
    \item Due to the law of large numbers, the price level satisfies:
\end{itemize}

$$
\left(P_{H, t}\right)^{1-\epsilon}=\phi\left(P_{H, t-1}\right)^{1-\epsilon}+(1-\phi)\left(\tilde{P}_{H, t}\right)^{1-\epsilon}
$$

\end{frame}

\begin{frame}{Market Clearing}

\begin{itemize}
    \item Goods market clearing requires that total production by the home firms is split between supply to the home and foreign markets respectively and satisfies the local demand in each market.
\end{itemize}


\begin{equation}
Y_t=C_{Ht}+C_{Ht}^*
\end{equation}

\begin{equation}
B_{t+1}/R_t-B_t=NX_t
\end{equation}

\begin{equation}
NX_t=\varepsilon_t P_{Ht}^*C_{Ht}^*-P_{Ft}*C_{Ft}
\end{equation}

\begin{itemize}
    \item The monetary policy in both countries is implemented via a conventional Taylor rule. 
\end{itemize}

\end{frame}




\section*{Reference}

\begin{frame}[allowframebreaks]
	\frametitle{References}
	\bibliographystyle{apalike}
	\footnotesize
	\bibliography{ref.bib}
\end{frame}

\end{document}