\documentclass[10pt]{beamer}

\usepackage{amssymb,amsthm}% http://ctan.org/pkg/amssymb
\newtheorem{proposition}{Proposition}
\setbeamertemplate{theorems}[numbered]
\usecolortheme[RGB={100,0,0}]{structure}
\setbeamercolor{block title}{use=structure,fg=white,bg=structure.fg!75!black}
\setbeamercolor{block body}{parent=normal text,use=block title,bg=block title.bg!10!bg}

\usepackage{amsmath}
\DeclareMathOperator*{\argmax}{argmax}
\DeclareMathOperator*{\argmin}{argmin}

\usepackage{natbib}
\bibliographystyle{aea}

\usepackage{xcolor}
\usepackage{graphicx}
\usepackage{amsmath}
\usepackage{numprint}
\npdecimalsign{.}
\nprounddigits{3}
\usepackage{colortbl}
\usepackage{appendixnumberbeamer}
\usepackage{subfigure}
\usepackage{comment}

%\usepackage[hyperref]{beamerarticle} or 
\usepackage{hyperref}
\usepackage{caption}
%\captionsetup{font=scriptsize,labelfont=scriptsize}
\setbeamerfont{caption}{size=\scriptsize}
\usetheme{Singapore}
\usecolortheme{beaver}

% for adding regression tables
\usepackage{dcolumn} 
% column to line up decimals
\usepackage{booktabs,caption}
\captionsetup[table]{name=Table} 
\usepackage[flushleft]{threeparttable} 
% The above two allow that last line with the dagger as a bottom note.
\captionsetup{skip=0pt}

\usepackage{xcolor}
\definecolor{underbrace}{RGB}{30,199,166}
\newcommand{\textfrac}[1]{
  \begin{tabular}{@{}l@{}}#1\end{tabular}
}

\begin{document}

\title{Global Monetary Policy Shock, Financial Friction and Export Price}
\author[Yao Amber LI, Lingfei LU and Jingbo YAO]{Yao Amber LI, Lingfei LU, Jingbo YAO}
\institute{\small HKUST}
\date{Dec 1, 2023}
\logo{\includegraphics[scale=0.15]{UST4C_L3.eps}}

\begin{frame}
\maketitle
\centering
HKUST Research Postgraduate Student Workshop
\end{frame}

\begin{frame}
\frametitle{Outline}
\tableofcontents
\end{frame}

%%%%%%%%%%%%%%%%%%%%%%%%%%%%%%%%%%%%%%%%%%%%%%%%%%%%%%
%%%%%%%%%%%%%%%%%%%%%%%%%%%%%%%%%%%%%%%%%%%%%%%%%%%%%%
\section{I. Introduction}

\begin{frame}{Motivation}
\begin{itemize}
    \item How export price responds to global shocks is essential in international economics
    \medskip
    \item Financial friction affects firms' exporting behavior
    \medskip
    \item International monetary shocks are the main driver of the global financial cycle
    \medskip
    \item  In an era of growing global financial and trade integration, understanding how global monetary shocks affect export prices by affecting firms' financial conditions is quite important
    \medskip
\end{itemize}
\end{frame}

\begin{frame}{This paper}
\begin{itemize}
    \item Using high-frequency exogenous monetary shocks, monthly custom transaction records, and comprehensive firm-level balance sheet data
    \medskip
    \item  A tightening US monetary policy shock significantly increases China's export prices through both the borrowing cost and liquidity channels.
    \medskip
    \item This impact is more pronounced for firms facing higher borrowing costs and tighter liquidity conditions. 
    \medskip
    \item A classical trade model that incorporates financial friction and external monetary policy shocks.
\end{itemize}
\end{frame}


\begin{frame}{Literature}
\begin{itemize}
    \item Financial friction and international trade
    \begin{itemize}
        \item Sector-specific and time-invariant constraint: \cite{manova2013credit}, \cite{lin2018international}
        \item \textcolor{red}{This paper: firm-level time-variant constraint and the role of global monetary shocks}
    \end{itemize}
    \item Determinants of export prices
        \begin{itemize}
        \item Exchange rate shocks: \cite{amiti2014importers}, \cite{auer2018quality}.
        \item Firm characteristics, destinations, and trade liberalization: \cite{manova2012export}, \cite{fan2015trade}, etc.
        \item \textcolor{red}{This paper: global monetary policy shocks}
    \end{itemize}
    \item Global monetary policy spillover
        \begin{itemize}
        \item Real economy: \cite{bluedorn2011open}, \cite{di2023impact}
        \item Asset prices: \cite{rogers2014evaluating}, \cite{miranda2020us}.
        \item \textcolor{red}{This paper: exporter pricing behaviors}
    \end{itemize}
\end{itemize}

\end{frame}

%%%%%%%%%%%%%%%%%%%%%%%%%%%%%%%%%%%%%%%%%%%%%%%%%%%%%%
%%%%%%%%%%%%%%%%%%%%%%%%%%%%%%%%%%%%%%%%%%%%%%%%%%%%%%
\section{II. Data}

\begin{frame}{Data: Monetary Policy Shock}
    \begin{itemize}
        \item Our US monetary policy shocks refer to \cite{bu2021unified}.
        \begin{itemize}
            \item This measure uses \cite{fama1973risk} two-step partial-least squares estimation using FOMC announcements.
        \end{itemize}
        \item This measure has several advantages:
        \begin{enumerate}
            \item Unpredictable from past available information (exogenous);
            \item No significant information effect (pure policy shock);
            \item Bridge conventional and unconventional monetary policy regimes
        \end{enumerate}
        \item We also use alternative measures for monetary policy shocks from \cite{nakamura2018high}, \cite{guraynak2005actions}, and \cite{jarocinski2020deconstructing}.
        \item The whole series is from 1994 to 2022, but we will focus on the period from 2000 to 2006 (84 months) within the matched sample.
    \end{itemize}
\end{frame}

\begin{frame}{Data: Monetary Policy Shock}
    \begin{figure}[htbp]
	\centering
	\includegraphics[width=0.9\columnwidth]{latex/drafts/pic/BRW.png}
	\label{sum.brw}
    \end{figure}
\end{frame}

\begin{frame}{Data: Firm-level and Customs data}
    \begin{itemize}
	\item First, we use annual surveys of Chinese manufacturing firms conducted by the National Bureau of Statistics of China (NBSC).
	\begin{itemize}
		\item This dataset covers all state-owned enterprises and above-scale firms with annual sales of more than 5 million RMB from 1999 to 2007. 
	\end{itemize}
        \item Our research will focus on the variables related to two aspects:
        \begin{enumerate}
            \item Production costs and sales, including total wage payment, total operation inputs, sales income, etc.
            \item Financial costs and liquidity, including financial expense and interest payment, total and net working capital, and cash holding.
        \end{enumerate}
        \item Then, we use monthly trade transaction records from the General Administration of Customs of China (GACC) from 2000-2006.
        \begin{itemize}
            \item This dataset provides universal information on Chinese trade transactions, including import and export values, quantities, product names and codes, source and destination countries, and firm types.
	\end{itemize}
         
    \end{itemize}
\end{frame}

\begin{frame}{Data: Summary Statistics}
    \begin{table}[htbp]
        \centering
        \caption{Summary statistics}
        \resizebox{0.9\columnwidth}{!}{
        \begin{threeparttable}
            \begin{tabular}{lccccc}
            \toprule
                    &        Mean&          SD&         p50&         p25&         p75\\
            \midrule
            $\Delta P$        &        0.03&        0.42&        0.01&       -0.11&        0.17\\
            \# HS6 Products           &        6.29&       10.31&        3.00&        2.00&        7.00\\
            Export Value (*1000 RMB)    &   129580&   372029&    19709&     3890&    80525\\
            Sales (*1000 RMB)   &   160148&  1201262&    34910&    15350&    90852\\
            Employment     &      449&     1210&      197&       96&      418\\
            $Debt$              &        0.55&        0.26&        0.56&        0.37&        0.74\\
            $IE/L$               &        0.02&        0.04&        0.01&        0.00&        0.03\\
            $Liquid$       &        0.10 &        0.29 &        0.10 &       -0.07 &        0.28\\
            $\phi^{exp}$ (Export/Sales)         &        0.46&        0.38&        0.36&        0.07&        0.89\\
            $\phi^{imp}$ (Export/Inputs)               &        0.18&        0.30&        0.01&        0.00&        0.25\\
            \midrule
            Firm-year observations        &     \multicolumn{5}{c}{270271}      \\
            \bottomrule
            \end{tabular}
            \begin{tablenotes}
        	\footnotesize
                \item Notes: This table shows the summary statistics of firms in the matched sample. The first row $\Delta P$ indicates monthly price changes, while all other rows describe annual-level firm variables. $Debt$ denotes the ratio of total liability over total assets, $IE/L$ denotes the ratio of interest expense over total liability, and $Liquid$ denotes the ratio of net liquid assets over total assets.
            \end{tablenotes}
        \end{threeparttable}
        }
        \label{tab.summary}
    \end{table}
\end{frame}

\begin{frame}{Export price index}
    \begin{itemize}
	\item We compute the unit values of each firm-product-country-year observation as the proxy of export prices:
        $$
        P_{chit}=\frac{V_{chit}\cdot NER_{US,t}}{Q_{chit}}
        $$
        \item We construct the firm-level Tornqvist price index as in \cite{smeets2013estimating}:
        \begin{itemize}
            \item First, we aggregate the unit value to the firm-product level price, $P_{hit}=\sum_c s_{c,hit} P_{chit}$, where the value share is $s_{c,hit}=V_{chit}/V_{hit}$.
            \item Second, we calculate the weighted average firm-product level price growth rate $\Delta_n \ln P_{hit} = \ln P_{hit}- \ln P_{hi(t-n)}$ across $n$ periods.
            \item Then, we calculate the firm-level price change as: 
            $$
            \Delta_n \ln P_{it} = \sum _{h} \frac{s_{h,i(t-n)}+s_{h,it}}{2} \Delta_n \ln P_{hit}
            $$
            where the product-specific value share is $s_{h,it}=V_{hit}/V_{it}$.
            \item In the monthly regression, we set the time gap $n$ as 12 months.
        \end{itemize}
    \end{itemize}
\end{frame}

%%%%%%%%%%%%%%%%%%%%%%%%%%%%%%%%%%%%%%%%%%%%%%%%%%%%%%
%%%%%%%%%%%%%%%%%%%%%%%%%%%%%%%%%%%%%%%%%%%%%%%%%%%%%%
\section{III. Empirical Results}

\begin{frame}{Benchmark specification}
    \begin{itemize}
        \item Our primary target is to study the impacts of monetary policy shocks on export prices.
        \item In our benchmark empirical analysis, we use the below specifications:
            \begin{equation}
                \Delta \ln P_{it} = \alpha+\beta \cdot m_{t}+ \Gamma \cdot \textbf{Z}_{it-1}+\xi_{i}+\varepsilon_{it}
            \end{equation}
        \begin{itemize}
            \item $\Delta \ln P_{it}$: the year-over-year export price index change;
            \item $m_t$: the unexpected monetary policy shock at time $t$;
            \item $\textbf{Z}_{it-1}$: firm-level (time-variant) lagged control terms;
            \item $\xi_{i}$: firm-level (time-invariant) fixed effects.
            \item $\beta$: the average export price response to the concurrent monetary policy surprises.
        \end{itemize}
        
    \end{itemize}
\end{frame}

\begin{frame}{Export price responses to monetary policy shocks}
    \begin{table}[htbp]
        \centering
        \caption{Export price responses to US monetary policy shocks}
        \resizebox{\columnwidth}{!}{
        \begin{threeparttable}
        \begin{tabular}{lcccccc}
            \toprule
            & (1)   & (2)   & (3)   & (4)   & (5)   & (6) \\
            & \multicolumn{3}{c}{Monthly $\Delta ln P_{it}$} & \multicolumn{3}{c}{Annual $\Delta ln P_{it}$}  \\
            \midrule
            $brw_t$   & 0.181*** & 0.145*** & 0.147*** & 0.136*** & 0.173*** & 0.173*** \\
                  & (0.011) & (0.010) & (0.010) & (0.007) & (0.009) & (0.009) \\
            $\Delta ln P_{it-1}$ &       & 0.301*** & 0.299*** &       & -0.297*** & -0.298***  \\
                  &       & (0.003) & (0.003) &       & (0.005) & (0.005) \\
            $Sales_{it-1}$ &       &       & -0.004*** &       &       &  0.004*\\
                  &       &       & (0.001) &       &       &  (0.003)\\
            \midrule
            Firm FE & Yes   & Yes   & Yes   & Yes   & Yes   & Yes \\
            N     & 1100399 & 940015 & 917435 & 151542 & 96296 & 96296 \\
            \bottomrule
        \end{tabular}
            \begin{tablenotes}
                \footnotesize
                \item Notes: Robust standard errors clustered at the firm level;  *, **, and *** indicate significance at 10\%, 5\%, and 1\% levels. The dependent variables in columns (1)-(3) are changes in monthly price, while columns (4)-(6) are changes in annual price. All regressions include firm fixed effects.
    	\end{tablenotes}
        \end{threeparttable}
        }
        \label{tab.baseline}
    \end{table}
\end{frame}

\begin{frame}{Robustness checks: Alternative monetary policy shocks}
    \begin{table}[htbp]
        \centering
        \caption{Alternative monetary policy shocks}
        \resizebox{\columnwidth}{!}{
        \begin{threeparttable}
        \begin{tabular}{lcccccccc}
            \toprule
            & (1)   & (2)   & (3)   & (4)   & (5)   & (6) & (7)   & (8) \\
            & \multicolumn{2}{c}{$\Delta$ FFR} & \multicolumn{2}{c}{Nakamura \& Steinsson}  & \multicolumn{2}{c}{Acosta} & \multicolumn{2}{c}{Jarocinski \& Karadi}\\
            \midrule
            $\Delta FFR_t$ & 0.029*** & 0.025***&       &        &       &       &       &  \\
                  & (0.008) & (0.007)&       &        &       &       &       &  \\
            $NS_t$  &       &    & 0.152*** & 0.144***     &       &  \\
                  &       &   & (0.009) & (0.009)       &       &  \\
            $Target^{US}_t$ &       &       &       &       & 0.002*** & 0.002*** &       &  \\
                  &       &       &       &       & (0.000) & (0.000) &       &  \\
            $Path^{US}_t$  &       &       &       &       & 0.005*** & 0.004*** &       &  \\
                  &       &       &       &       & (0.000) & (0.000) &       &  \\
            $MP_t$ &       &       &       &       &       &       & 0.045*** & 0.068*** \\
                  &       &       &       &       &       &       & (0.006) & (0.006) \\
            $\Delta ln P_{it-1}$ &       & 0.299*** &       & 0.300*** &       & 0.299*** &       & 0.300*** \\
                  &       & (0.003) &       & (0.003) &       & (0.003) &       & (0.003) \\
            $Sales_{it-1}$ &       & -0.005*** &       & -0.004*** &       & -0.005*** &       & -0.004*** \\
                  &       & (0.001) &       & (0.001) &       & (0.001) &       & (0.001) \\
            \midrule
            Firm FE & Yes   & Yes   & Yes   & Yes   & Yes   & Yes   & Yes   & Yes \\
            N     & 1100399 & 917435 & 1100399 & 917435 & 1100399 & 917435 & 1100399 & 917435 \\
            \bottomrule
        \end{tabular}
            \begin{tablenotes}
                \footnotesize
                \item Notes: Robust standard errors clustered at the firm level;  *, **, and *** indicate significance at 10\%, 5\%, and 1\% levels. The dependent variables in all columns are changes in monthly price. Monetary policy shock measures in columns (1)-(2), (3)-(4), (5)-(6), (7)-(8) are from Fed fund rate changes, \cite{nakamura2018high}, \cite{acosta2022perceived}, and \cite{jarocinski2020deconstructing}, respectively. All regressions include firm fixed effects.
    	\end{tablenotes}
        \end{threeparttable}
        }
        \label{tab.altmps}
    \end{table}
\end{frame}

\begin{frame}[label=robustness_other]{Robustness checks: Other tests}
    In addition, we also adopt other robustness tests:
    \medskip
    \begin{enumerate}
        \item Alternative \textbf{aggregation levels of price index} \hyperlink{appendix_tab.altagg}{\beamergotobutton{Alternative aggregation}}
        \item Sub-sample with \textbf{only single product firms} \hyperlink{appendix_tab.single}{\beamergotobutton{Single product firms}}
        \item Alternative \textbf{standard error cluster levels and fixed effects} \hyperlink{appendix_tab.altfe}{\beamergotobutton{Alternative SE clusters and FEs}}
        \item Export prices \textbf{denominated in US dollars} \hyperlink{appendix_tab.USD}{\beamergotobutton{USD prices}}
    \end{enumerate}
\end{frame}

%%%%%%%%%%%%%%%%%%%%%%%%%%%%%%%%%%%%%%%%%%%%%%%%%%%%%%
%%%%%%%%%%%%%%%%%%%%%%%%%%%%%%%%%%%%%%%%%%%%%%%%%%%%%%

\section{IV. Mechanism}

\begin{frame}{Decomposition of prices: markup vs marginal cost}
    \begin{itemize}
        \item To study why prices adjust with monetary policy shocks, we further decompose export prices into markups and marginal costs.
        \begin{itemize}
            \item The markup is estimated as the ratio of an input factor's output elasticity to its firm-specific factor payment share $\mu_{t}=\theta_{t}^{X}\left(\alpha_{t}^{X}\right)^{-1}$
            \item We the structural assumptions of \cite{deloecker2012markups} and the GMM estimation by \cite{brooks2021agglomeration}.
        \end{itemize}
        \item We replace the price changes with the markup and marginal cost changes or add markups into the control terms. 
        \begin{equation}
            \Delta \mu_{it} = \alpha +\beta \cdot brw_{t}+ \Gamma \cdot \textbf{Z}_{it-1}+\xi_{i}+\varepsilon_{it} \label{reg.markup}
        \end{equation}
        \begin{equation}
            \Delta \ln P_{it} = \alpha+\beta \cdot brw_{t}+ \gamma \cdot \Delta \mu_{it}+ \Gamma \cdot \textbf{Z}_{it-1}+\xi_{i}+\varepsilon_{i t} \label{reg.markup_int}
        \end{equation}
    \end{itemize}
\end{frame}

\begin{frame}{Decomposition of prices: markup vs marginal cost}
\begin{table}[htbp]
    \centering
    \caption{Decomposition of prices: markup vs marginal cost}
    \resizebox{\columnwidth}{!}{
    \begin{threeparttable}
    \begin{tabular}{lcccccccc}
        \toprule
        & (1)   & (2)   & (3)   & (4)   & (5)   & (6) & (7)   & (8) \\
        & \multicolumn{4}{c}{Monthly sample} & \multicolumn{4}{c}{Annual sample}  \\
        \cline{2-5}  \cline{6-9}
        & $\Delta Markup_{it}$ & $\Delta ln(MC)_{it}$ & \multicolumn{2}{c}{$\Delta ln P_{it}$}  & $\Delta Markup_{it}$ & $\Delta MC_{it}$ & \multicolumn{2}{c}{$\Delta ln P_{it}$}\\
        \midrule
        $brw_t$   & -0.010 & 0.190*** & 0.149*** & 0.036*** & -0.017** & 0.174*** & 0.176*** & 0.084*** \\
              & (0.008) & (0.015) & (0.012) & (0.007) & (0.009) & (0.012) & (0.010) & (0.007) \\
        $\Delta Markup_{it}$&       &       & 0.004* &       &       &       &   0.007*    &  \\
              &       &       & (0.002) &       &       &       &   (0.004)    &  \\
        $\Delta MC_{it}$&       &       &       & 0.668*** &       &       &       &  0.472***\\
              &       &       &       & (0.004) &       &       &       &  (0.005)\\
        $\Delta ln P_{it-1}$&       &       & 0.280*** & 0.099*** &       &       &   -0.307*** & -0.163*** \\
              &       &       & (0.003) & (0.002) &       &       &   (0.005) & (0.004)  \\
        $Sales_{it-1}$ & -0.022*** & 0.019*** & -0.005*** & -0.018*** &  -0.022*** & 0.018*** & 0.006** & -0.009*** \\
              & (0.003) & (0.004) & (0.002) & (0.002) &  (0.002) & (0.003) & (0.003) & (0.002) \\
        \midrule
        Firm FE & Yes   & Yes   & Yes   & Yes   & Yes   & Yes   & Yes   & Yes \\
        N     & 830585 & 767768 & 665438 & 661680 & 110750 & 105330 & 81471 & 80831 \\
        \bottomrule
    \end{tabular}
        \begin{tablenotes}
            \footnotesize
            \item Notes: Robust standard errors clustered at the firm level;  *, **, and *** indicate significance at 10\%, 5\%, and 1\% levels. The dependent variables in columns (1), (2), (3)-(4) are changes in markup, marginal cost, and price in the monthly sample; the dependent variables in columns (5), (6), (7)-(8) are changes in markup, marginal cost, and price in the annual sample. All regressions include firm fixed effects.
	\end{tablenotes}
    \end{threeparttable}
    }
    \label{tab.markup}
\end{table}
\end{frame}

\begin{frame}{Borrowing cost and liquidity conditions}
    \begin{itemize}
        \item Two hypotheses about the mechanism of monetary policy effects:
        \begin{enumerate}
            \item \textbf{Global borrowing cost channel}. A tightening US monetary shock deteriorates the financing environment and increases the borrowing cost of firms, forcing them to offset this cost by lifting export prices.
            \item \textbf{Global liquidity channel}. A contractionary US monetary shock aggregates firms' liquidity conditions, which induces firms to raise prices to alleviate liquidity constraints.
        \end{enumerate}
        \item We will check whether and how borrowing costs and liquidity conditions of firms change with the monetary policy shocks:
        \begin{equation}
            \Delta F_{it} = \alpha +\beta \cdot brw_{t}+ \Gamma \cdot \textbf{Z}_{it-1}+\xi_{i}+\varepsilon_{it} \label{reg.liquid}
        \end{equation}
    \end{itemize}
\end{frame}

\begin{frame}{Borrowing cost and liquidity conditions}
    \begin{table}[htbp]
        \centering
        \caption{Borrowing cost channel and liquidity channel}
        \resizebox{\columnwidth}{!}{
        \begin{threeparttable}
        \begin{tabular}{lcccccccc}
            \toprule
            & (1)   & (2)   & (3)   & (4)   & (5)   & (6) & (7)   & (8) \\
            & $\Delta IE/L$ & $\Delta IE/CL$ & $\Delta FN/L$ & $\Delta FN/CL$ & $\Delta WC$  & $\Delta Liquid$ & $\Delta Cash$ & $\Delta Arec$ \\
            \midrule
            $brw_t$   & 0.002*** & 0.004*** & 0.002*** & 0.004*** & -0.005*** & -0.021*** & -0.007*** & -0.004** \\
                  & (0.000) & (0.001) & (0.001) & (0.001) & (0.002) & (0.002) & (0.002) & (0.002) \\
            $Sales_{it-1}$ & -0.001*** & -0.002*** & -0.003*** & -0.004*** & -0.009*** & -0.000 & -0.001*** & 0.059*** \\
                  & (0.000) & (0.000) & (0.000) & (0.000) & (0.000) & (0.001) & (0.000) & (0.001) \\
            $Debt_{it-1}$ & 0.045*** & 0.052*** & 0.080*** & 0.089*** & -0.105*** & 0.599*** & -0.005*** & -0.053*** \\
                  & (0.001) & (0.001) & (0.001) & (0.001) & (0.002) & (0.002) & (0.002) & (0.002) \\
            \midrule
            Firm FE & Yes   & Yes   & Yes   & Yes   & Yes   & Yes   & Yes   & Yes \\
            N     & 1081803 & 1056283 & 1081803 & 1056283 & 1090620 & 1090620 & 1090620 & 1090620 \\
            \bottomrule
        \end{tabular}
            \begin{tablenotes}
                \footnotesize
                \item Notes: Robust standard errors clustered at the firm level;  *, **, and *** indicate significance at 10\%, 5\%, and 1\% levels. The dependent variables in columns (1)-(4) are changes in interest expense over the total liability ratio, interest expense over the current liability ratio, total financial expense over the total liability ratio, and total financial expense over the current liability ratio, respectively. The dependent variables in columns (5)-(8) are changes in working capital over total asset ratio, net liquidity asset over total asset ratio, cash over total asset ratio, and accounts receivable over sales ratio, respectively. All regressions include firm fixed effects.
    	\end{tablenotes}
        \end{threeparttable}
        }
        \label{tab.liquid}
    \end{table}
\end{frame}

\begin{frame}{Confounding effects of other production costs}
    \begin{table}[htbp]
        \centering
        \caption{Discussion about other production costs}
        \resizebox{0.9\columnwidth}{!}{
        \begin{threeparttable}
        \begin{tabular}{lccccc}
            \toprule
            & (1)   & (2)   & (3)   & (4)   & (5) \\
            & $\Delta Input/Sales$ & $\Delta Wage/Sales$ & \multicolumn{3}{c}{$\Delta ln P_{it}$}\\
            \midrule
            $brw_t$   & 0.403 & -0.164 & 0.142*** & 0.159*** & 0.154*** \\
                  & (0.263) & (0.135) & (0.011) & (0.013) & (0.014) \\
            $brw_t \times Input/Sales $ &       &       & 0.007 &       &       \\
                  &       &       & (0.005) &       &      \\
            $brw_t \times Wage/Sales $ &       &       &       & -0.108 &         \\
                  &       &       &       & (0.078) &     \\
            $brw_t \times \phi^{imp}_t$ &       &       &       &       & -0.027  \\
                  &       &       &       &       & (0.032)  \\
            $Debt_{it_1}$ & 0.436** & 0.241 &       &       &       \\
                  & (0.180) & (0.162) &       &       &    \\
            $\Delta ln P_{it-1}$ &       &       & 0.299*** & 0.299*** & 0.299***  \\
                  &       &       & (0.003) & (0.003) & (0.003)  \\
            $Sales_{it-1}$ &  1.075*** & -0.044   & -0.004*** & -0.004*** & -0.004*** \\
                  &  (0.262) & (0.187) & (0.001) & (0.001) & (0.001) \\
            \midrule
            Firm FE & Yes   & Yes   & Yes   & Yes   & Yes \\
            N     & 1090620 & 1090620 & 917435 & 917435 & 917435 \\
            \bottomrule
        \end{tabular}
            \begin{tablenotes}
                \footnotesize
                \item Notes: Robust standard errors clustered at the firm level;  *, **, and *** indicate significance at 10\%, 5\%, and 1\% levels. The dependent variables in columns (1)-(2) are changes in intermediate input cost over sales ratio and wage expense over sales ratio, respectively. The dependent variables in columns (3)-(7) are changes in monthly price. All regressions include firm fixed effects.
    	\end{tablenotes}
        \end{threeparttable}
        }
        \label{tab.other}
    \end{table}
\end{frame}

\begin{frame}{Non-linear effects of borrowing cost and liquidity}
    \begin{table}[htbp]
        \centering
        \caption{Non-linearity of borrowing cost channel and liquidity channel}
        \resizebox{0.85\columnwidth}{!}{
        \begin{threeparttable}
        \begin{tabular}{lcccccccc}
            \toprule
            & (1)   & (2)   & (3)   & (4)   & (5)   & (6)   & (7)   & (8) \\
        \midrule
        $brw_t$ & 0.107*** & 0.109*** & 0.087*** & 0.092*** & 0.351*** & 0.236*** & 0.510*** & 0.129*** \\
              & (0.017) & (0.017) & (0.033) & (0.031) & (0.097) & (0.022) & (0.082) & (0.025) \\
        $brw_t \times IE/L_{st}$ & 6.580*** &       &       &       &       &       &       &  \\
              & (1.977) &       &       &       &       &       &       &  \\
        $brw_t \times IE/CL_{st}$ &       & 5.415*** &       &       &       &       &       &  \\
              &       & (1.701) &       &       &       &       &       &  \\
        $brw_t \times FN/L_{st}$ &       &       & 4.203** &       &       &       &       &  \\
              &       &       & (2.112) &       &       &       &       &  \\
        $brw_t \times FN/CL_{st}$ &       &       &       & 3.466** &       &       &       &  \\
              &       &       &       & (1.768) &       &       &       &  \\
        $brw_t \times WC_{st}$ &       &       &       &       & -0.331** &       &       &  \\
              &       &       &       &       & (0.159) &       &       &  \\
        $brw_t \times Liquid_{st}$ &       &       &       &       &       & -1.137*** &       &  \\
              &       &       &       &       &       & (0.270) &       &  \\
        $brw_t \times Cash_{st}$ &       &       &       &       &       &       & -2.226*** &  \\
              &       &       &       &       &       &       & (0.498) &  \\
        $brw_t \times Arec_{st}$ &       &       &       &       &       &       &       & 0.167 \\
              &       &       &       &       &       &       &       & (0.218) \\
        $\Delta ln P_{it-1}$ & 0.299*** & 0.299*** & 0.299*** & 0.299*** & 0.299*** & 0.299*** & 0.299*** & 0.299*** \\
              & (0.003) & (0.003) & (0.003) & (0.003) & (0.003) & (0.003) & (0.003) & (0.003) \\
        $Sales_{it-1}$ & -0.004*** & -0.004*** & -0.004*** & -0.004*** & -0.004*** & -0.004*** & -0.004*** & -0.004*** \\
              & (0.001) & (0.001) & (0.001) & (0.001) & (0.001) & (0.001) & (0.001) & (0.001) \\
        \midrule
        Firm FE & Yes   & Yes   & Yes   & Yes   & Yes   & Yes   & Yes   & Yes \\
        N     & 917435 & 917435 & 917435 & 917435 & 917435 & 917435 & 917435 & 917435 \\
            \bottomrule
        \end{tabular}
            \begin{tablenotes}
                \footnotesize
                \item Notes: Robust standard errors clustered at the firm level;  *, **, and *** indicate significance at 10\%, 5\%, and 1\% levels. The interaction terms in columns (1)-(8) are the industry-level average interest expense over the total liability ratio, interest expense over the current liability ratio, total financial expense over the total liability ratio, total financial expense over the current liability ratio, working capital over total asset ratio, net liquidity asset over total asset ratio, cash over total asset ratio, and accounts receivable over sales ratio. All regressions include firm fixed effects.
    	\end{tablenotes}
        \end{threeparttable}
        }
        \label{tab.non-linearity}
    \end{table}    
\end{frame}

\begin{frame}{Ordinary trade vs processing trade}
    \begin{itemize}
        \item \textbf{Processing trade}: a firm imports raw materials and intermediate inputs from a foreign firm for domestic processing and re-exports to the same firm. It is less dependent on external financing.
    \end{itemize}
    \begin{table}[h]
        \centering
        \caption{Ordinary trade vs processing trade}
        \resizebox{0.85\columnwidth}{!}{
        \begin{threeparttable}
        \begin{tabular}{lcccccc}
            \toprule
            & (1)   & (2)   & (3)   & (4)   & (5)   & (6) \\
              & \multicolumn{2}{c}{Ordinary trade}  & \multicolumn{2}{c}{Processing trade} & \multicolumn{2}{c}{Comparison}\\
            \midrule
            $brw_t$   & 0.201*** & 0.186*** & 0.098*** & 0.065*** & 0.225*** & 0.185*** \\
                  & (0.018) & (0.019) & (0.019) & (0.016) & (0.016) & (0.016) \\
            $brw_t \times process_{it}$ &       &       &       &       & -0.102*** & -0.085*** \\
                  &       &       &       &       & (0.024) & (0.023) \\
            $\Delta ln P_{it-1}$ &       & 0.190*** &       & 0.473*** &       & 0.299*** \\
                  &       & (0.003) &       & (0.005) &       & (0.003) \\
            $Sales_{it-1}$ &       & -0.004* &       & -0.009*** &       & -0.004*** \\
                  &       & (0.002) &       & (0.002) &       & (0.001) \\
            Firm FE & Yes   & Yes   & Yes   & Yes   & Yes   & Yes \\
            N     & 499418 & 391336 & 283952 & 242587 & 1100399 & 917435 \\
            \bottomrule
        \end{tabular}
            \begin{tablenotes}
                \footnotesize
                \item Notes: Robust standard errors clustered at the firm level;  *, **, and *** indicate significance at 10\%, 5\%, and 1\% levels. The dependent variables in columns (1)-(3) are changes in monthly price denominated in the US dollar, while columns (4)-(6) are changes in annual price denominated in the US dollar. All regressions include firm fixed effects.
    	\end{tablenotes}
        \end{threeparttable}
        }
        \label{tab.process}
    \end{table}
\end{frame}

\begin{frame}{Alternative explanations}
    \begin{itemize}
        \item \textbf{Demand shift}: Will global demand shift to Chinese goods in the global recession caused by tightening global monetary shock because of the lower quality?
        \begin{itemize}
            \item \textbf{Homogeneous}: reference priced or traded on an organized exchange.
            \item \textbf{Differentiated}: each firm sets a different price for its product.
        \end{itemize}
    \end{itemize}
    \begin{table}[htbp]
        \centering
        \caption{Homogeneous good vs differentiated good}
        \resizebox{0.9\columnwidth}{!}{
        \begin{threeparttable}
        \begin{tabular}{lcccccccc}
            \toprule
            & (1)   & (2)   & (3)   & (4)   & (5)   & (6)   & (7)   & (8) \\
            & \multicolumn{4}{c}{Conservative classification} & \multicolumn{4}{c}{Liberal classification} \\
            \midrule
            $brw_t$   & 0.170*** & 0.126*** & 0.149*** & 0.114*** & 0.168*** & 0.124*** & 0.148*** & 0.115*** \\
              & (0.011) & (0.011) & (0.012) & (0.012) & (0.012) & (0.011) & (0.013) & (0.012) \\
            $brw_t \times ToE_{it} $ & 0.152 & 0.112 &       &    & 0.263*** & 0.242***    &       &  \\
              & (0.128) & (0.125) &       &      & (0.086) & (0.082)     &       &  \\
            $brw_t \times Ref_{it} $ &       &       & 0.208*** & 0.125*** &       &    & 0.167*** & 0.082*** \\
              &       &       & (0.033) & (0.032) &       &         & (0.031) & (0.030)\\
            $\Delta ln P_{it-1}$ &       & 0.298*** &       & 0.298*** &       & 0.298*** &       & 0.298*** \\
              &       & (0.003) &       & (0.003) &       & (0.003) &       & (0.003) \\
            $Sales_{it-1}$ &       & 0.004*** &       & 0.004*** &       & 0.004*** &       & 0.004*** \\
              &       & (0.001) &       & (0.001) &       & (0.001) &       & (0.001) \\
        \midrule
        Firm FE & Yes   & Yes   & Yes   & Yes   & Yes   & Yes   & Yes   & Yes \\
        N     & 1014106 & 850174 & 1014106 & 850174 & 1014106 & 850174 & 1014106 & 850174 \\
            \bottomrule
        \end{tabular}
            \begin{tablenotes}
                \footnotesize
                \item Notes: Robust standard errors clustered at the firm level;  *, **, and *** indicate significance at 10\%, 5\%, and 1\% levels. The variables "ToE" and "Ref" represent the value share of goods traded on an organized exchange and the value share of reference-priced goods of firm $i$. Columns (1)-(4) use the "conservative" classification, while columns (5)-(8) use the "liberal" classification, both referring to \cite{rauch1999networks}. All regressions include firm fixed effects.
    	\end{tablenotes}
        \end{threeparttable}
        }
        \label{tab.rauch}
    \end{table}
\end{frame}

\begin{frame}[label=alt_erpt]{Alternative explanations}
    \begin{itemize}
        \item \textbf{Exchange rate pass-through}: A tightening US monetary shock will induce the depreciation of RMB, which will increase export prices denominated in RMB.
        \item However, this explanation is not likely to be the main reason:
        \begin{enumerate}
            \item In most of the sample period, China's exchange rate regime is fixed to the US dollar. US tightening will cause the RMB to depreciate against other third-country currencies.
            \item We control the change of bilateral real exchange rate in the annual firm-product-country level regression, and the results are consistent. \hyperlink{appendix_tab.altagg}{\beamergotobutton{Alternative aggregation}}
            \item The average export exchange rate pass-through of Chinese firms is near complete (about 95\%), as in \cite{li2015exchange}.
        \end{enumerate}
        
    \end{itemize}
\end{frame}

%%%%%%%%%%%%%%%%%%%%%%%%%%%%%%%%%%%%%%%%%%%%%%%%%%%%%%
%%%%%%%%%%%%%%%%%%%%%%%%%%%%%%%%%%%%%%%%%%%%%%%%%%%%%%

\section{V. More Discussion}

\begin{frame}{EU monetary policy shocks}
    \begin{itemize}
        \item The European Central Bank (ECB) is another major central bank.
        \item What if we replace US Fed shocks with ECB shocks?
    \end{itemize}
    \begin{table}[htbp]
        \centering
        \caption{Export price responses to EU monetary policy shocks}
        \resizebox{0.9\columnwidth}{!}{
        \begin{threeparttable}
        \begin{tabular}{lcccccc}
            \toprule
            & (1)   & (2)   & (3)   & (4)   & (5)   & (6) \\
            & \multicolumn{3}{c}{Miranda-Agrippino \& Nenova} & \multicolumn{3}{c}{Jarocinski \& Karadi}  \\
            \midrule
            $m^{EU}_t$ & -0.002*** & -0.001*** & -0.001*** & -0.050*** & 0.002 & -0.002 \\
                  & (0.000) & (0.000) & (0.000) & (0.011) & (0.011) & (0.011)\\  
            $\Delta ln P_{it-1}$ &       & 0.301*** & 0.300*** &       & 0.301*** & 0.300*** \\
                  &       & (0.003) & (0.003) &       & (0.005) & (0.005) \\
            $Sales_{it-1}$ &       &       & -0.004*** &       &       & -0.004*** \\
                  &       &       & (0.001) &       &       & (0.001) \\
            \midrule
            Firm FE & Yes   & Yes   & Yes   & Yes   & Yes   & Yes \\
             N     & 1100399 & 940015 & 917435 & 1100399 & 940015 & 917435 \\
            \bottomrule
        \end{tabular}
        \begin{tablenotes}
            \footnotesize
            \item Notes: Robust standard errors clustered at the firm level;  *, **, and *** indicate significance at 10\%, 5\%, and 1\% levels. Monetary policy shocks here are from the European Central Bank. The shock in columns (1)-(3) are obtained from \cite{miranda2022tale}, while columns (4)-(6) are  from \cite{jarocinski2020deconstructing}. All regressions include firm fixed effects.
    	\end{tablenotes}
        \end{threeparttable}
        }
        \label{tab.EU}
    \end{table}
    \begin{itemize}
        \item This indicates the special role of US policy as a main driver of the global financial cycle.
    \end{itemize}
\end{frame}

\begin{frame}{Exchange rate regime shift: fixed VS floating}
    \begin{itemize}
        \item After July 2005, China's exchange rate regime shifted from the US dollar peg to targeting a basket of currencies (semi-floating).
        \item Floating exchange rate regime acts as a buffer to alleviate adverse shock.
    \end{itemize}
    \begin{table}[htbp]
        \centering
        \caption{Exchange rate regime shift: before and after July 2005}
        \resizebox{0.9\columnwidth}{!}{
        \begin{threeparttable}
        \begin{tabular}{lcccccc}
            \toprule
            & (1)   & (2)   & (3)   & (4)   & (5)   & (6) \\
            & \multicolumn{3}{c}{Fixed regime: before July 2005} & \multicolumn{3}{c}{Floating regime: after July 2005}  \\
            \midrule
            $brw_t$   & 0.189*** & 0.138*** & 0.138*** & 0.031 & 0.108*** & 0.080*** \\
                  & (0.012) & (0.012) & (0.012) & (0.022) & (0.023) & (0.023) \\
            $\Delta ln P_{it-1}$ &       & 0.288*** & 0.287*** &       & 0.186*** & 0.182*** \\
                  &       & (0.003) & (0.003) &       & (0.004) & (0.004) \\
            $Sales_{it-1}$ &       &       & 0.000 &       &       & -0.025*** \\
                  &       &       & (0.002) &       &       & (0.004) \\
            \midrule
            Firm FE & Yes   & Yes   & Yes   & Yes   & Yes   & Yes \\
            N     & 708254 & 609726 & 594891 & 389776 & 328076 & 320432 \\
            \bottomrule
        \end{tabular}
            \begin{tablenotes}
                \footnotesize
                \item Notes: Robust standard errors clustered at the firm level;  *, **, and *** indicate significance at 10\%, 5\%, and 1\% levels. Columns (1)-(3) cover the period from January 2000 to July 2005, while columns (4)-(6) cover the period from August 2005 to December 2006. All regressions include firm fixed effects.
    	\end{tablenotes}
        \end{threeparttable}
        }
        \label{tab.regime}
    \end{table}
\end{frame}

%%%%%%%%%%%%%%%%%%%%%%%%%%%%%%%%%%%%%%%%%%%%%%%%%%%%%%
%%%%%%%%%%%%%%%%%%%%%%%%%%%%%%%%%%%%%%%%%%%%%%%%%%%%%%
\section{VI. Model}

\begin{frame}{Preferences and demand}

We introduce a general trade model as Meliz (2003) and Manova (2012). \\

The source and destination countries are denoted by $i$ and $j$, respectively. Consumers in the country $j$ can access a set of goods $X_j$. \\
\medskip
A representative consumer in country $j$ has a constant elasticity of substitution (CES) utility:

\begin{equation}
U_j=(\int_{\omega \in \Omega_j} [\chi_{ij}(\omega)]^{\frac{\sigma-1}{\sigma}} d\omega\ )^\frac{\sigma}{\sigma-1}
\end{equation}

where $\chi_{ij}$ is country $j$’s quantity consumed of variety $\omega$ originated from country $i$, and $\sigma$ $>$ 1 is the elasticity of substitution between varieties. 

\end{frame}



\begin{frame}{Preferences and demand}

Consumer optimization yields the following demand function for variety $\omega$:

\begin{equation}
\chi_{ij}(\omega)=\frac{p_{ij}(\omega)^{-\sigma}}{P_j^{-\sigma}} Y_j
\end{equation}


where $p_{ij}(\omega)$ is the price of variety $\omega$, $P_j=(\int_{\omega \in \Omega_j} [p_{ij}(\omega)]^{1-\sigma} d \omega)^{\frac{1}{1-\sigma}}$ is an aggregate price index in country $j$, and $Y_j$ represents the total expenditure of country $j$. \\

We assume that it is affected by global monetary shocks $m$ (e.g., US shock): $Y_j=\bar{Y_j}+\rho_{Y}^j m+\epsilon_Y$, where $\bar{Y_j}$ is a trend component of $Y_j$, $\rho_{Y}^j<0$, $\epsilon_Y$ is a random error, and a positive $m$ means a tightening shock.

\end{frame}



\begin{frame}{Exporting firm}

\textbf{Assumption}: working capital constraint
\vfill

A fraction $\delta_i$ of the input costs should be borrowed from outside financial institutions and paid in advance. Here $\delta_i$ $\in$ [0, 1] and is a decreasing function of the firm's liquidity condition and trade credit: 

$$
\delta_i \equiv 1-c_i^\gamma-\zeta_i
$$

where $c_i$ $\in$ [0, 1] and is the liquidity condition (bigger value means better situation), $\gamma$ is a positive constant and reflects the elasticity of borrowing fraction with respect to liquidity condition, $\zeta_i$ is the fraction of input costs that are supported by trade credit which is assumed to be affected by global monetary shocks $c_i=\bar{c_i}+\rho_c^i m+\epsilon_c^i$, $\rho_c^i<0$, $\zeta_i=\bar{\zeta_i}+\rho_\zeta^i m+\epsilon_t^i$,  $\rho_\zeta^i<0$. These impacts are supported by literature and our empirical evidence.

\end{frame}


\begin{frame}{Exporting firm}

\textbf{The production function}:

$$
y_i= \phi_i L_i
$$

where $\phi_i$ is productivity and $L_i$ is input.The firm in country $i$ minimizes its cost to satisfy the demand in the country $j$, $\chi_{ij}(\omega)=\frac{p_{ij}(\omega)^{-\sigma}}{P_j^{-\sigma}} Y_j$.
\vfill

\textbf{The cost function}:

$$
C_{ij}=\frac{w_i(1-\delta_i+\delta_i R_i)}{\phi_i} \frac{p_{ij}(\omega)^{-\sigma}}{P_j^{-\sigma}} Y_j
$$ 

$w_i$ is the price of input and $R_i$ is the gross borrowing interest rate in country $i$. We explicitly assume $R_i=\bar{R_i}+\rho_R^i m+\epsilon_R^i$, $\rho_R^i>0$. To capture the demand side effect, we allow the input price to fluctuate in response to global monetary shocks: $w_i=\bar{w_i}+\rho_w^i m + \epsilon_w^i$, $\rho_w^i<0$.

\end{frame}


\begin{frame}{Firm's problem}

We also assume that firms cannot borrow more than a fraction $\theta$ of the expected cash flow from exporting, and it is smaller when $R$ is higher. The intuition behind this is that higher interest rates imply higher risk, and access to credit should be lower. Without loss of generality, we can assume $\theta=R^{-\nu}$, where $\nu$ is $>0$ and reflects the elasticity of financial credit access with respect to the interest rate. The optimization problem of the firm is 

$$
\max_{p} \ (p- \frac{\tau w(1-\delta+\delta R^\alpha)}{\phi}) \frac{p^{-\sigma}}{P^{-\sigma}} Y
$$

\begin{equation}
\text{s.t.} \ \theta [(p-\zeta \frac{\tau w}{\phi}) \frac{p^{-\sigma}}{P^{-\sigma}} Y]\geq(1-c^\gamma-\zeta)\frac{\tau w}{\phi} \frac{p^{-\sigma}}{P^{-\sigma}} Y
\end{equation}

\end{frame}



\begin{frame}{Optimal price}

\textbf{Case 1: borrowing constraint is binding: }

If $\theta\leq\bar{\theta}$, where $\bar{\theta}$ is a threshold, the borrowing constraint is binding and we can rewrite the borrowing constraint as 

\begin{equation}
p=[(1-R^{\nu})\zeta+(1-c^\gamma)R^{\nu}] \frac{\tau w}{\phi}
\end{equation}

\textbf{Case 2: borrowing constraint is not binding}

If $\theta>\bar{\theta}$, the borrowing constraint is not binding. Solving the unconstrained optimization problem will give us:

\begin{equation}
p=\frac{\sigma}{\sigma-1}\frac{\tau w [c^\gamma+\zeta+(1-c^\gamma-\zeta) R^\alpha]}{\phi}
\end{equation}

This reduces to $p=\frac{\sigma}{\sigma-1}\frac{\tau w}{\phi}$ if $c$=0 and $R$=1, which is similar to \cite{melitz2003impact} 

\end{frame}


\begin{frame}{Proposition}

\textbf{Proposition 1.} The export price increases with the borrowing rate and decreases with liquidity condition and trade credit: $\frac{\partial p}{\partial R}>0$, $\frac{\partial p}{\partial c}<0$ $\frac{\partial p}{\partial \zeta}<0$
\vfill
\textbf{Proposition 2.} The export price increases after a tightening shock (i.e. $\frac{\partial p}{\partial m}>0$) if the supply side effect dominates.
\vfill
\textbf{Proposition 3.} The impact of monetary shock on export price (i.e. $\frac{\partial p}{\partial m}$) is non-linear. Conditional on the value of specific parameters, it is bigger when $R$ is larger and $c$ (or $\zeta$) is smaller.

\end{frame}

\begin{frame}{Proposition}

\textbf{Proof of Proposition 2}

If the borrowing constraint is binding:

\begin{align*} 
\frac{\partial p}{\partial m} =&\frac{\partial p}{\partial R}\frac{\partial R}{\partial m}+\frac{\partial p}{\partial c}\frac{\partial c}{\partial m}+\frac{\partial p}{\partial \zeta}\frac{\partial \zeta}{\partial m}+\frac{\partial p}{\partial w}\frac{\partial w}{\partial m}  \\
=& \frac{\tau w}{\phi}(1-c^\gamma-\zeta)\nu R^{\nu-1} \rho_R+ \frac{\tau w}{\phi} R^\nu(-1)\gamma c^{\gamma-1} \rho_c +\\  
& \frac{\tau w}{\phi}(1-R^\nu) \rho_t+\frac{\tau}{\phi}[(1-R^\nu)\zeta+(1-c^\gamma)R^\nu] \rho_w
\end{align*} 

If the borrowing constraint is not binding:

\begin{align*} 
\frac{\partial p}{\partial m} =&\frac{\partial p}{\partial R}\frac{\partial R}{\partial m}+\frac{\partial p}{\partial c}\frac{\partial c}{\partial m}+\frac{\partial p}{\partial \zeta}\frac{\partial \zeta}{\partial m}+\frac{\partial p}{\partial w}\frac{\partial w}{\partial m}  \\
=& \frac{\sigma}{\sigma-1}\frac{\tau w}{\phi}[\alpha(1-c^\gamma-\zeta)R^{\alpha-1}] \rho_R + \frac{\sigma}{\sigma-1}\frac{\tau w}{\phi}\gamma(1-R^\alpha)c^{\gamma-1} \rho_c + \\
& \frac{\sigma}{\sigma-1}\frac{\tau w}{\phi} (1-R^\alpha) \rho_\zeta + \frac{\sigma}{\sigma-1}\frac{\tau}{\phi}[c^\gamma+\zeta+(1-c^\gamma-\zeta)R^\alpha] \rho_w
\end{align*} 

\end{frame}



\begin{frame}[label=model_extension]{Model extension}
\fontsize{11}{11}\selectfont

Our conclusion is robust to:
\medskip

\begin{itemize}
    \item Dynamic optimization and sticky price
    \medskip
    \item Currency invoicing: DCP, LCP, PCP
    \medskip
    \item Both variable and fixed costs are paid in advance
\end{itemize}

\hyperlink{appendix_model_extension}{\beamergotobutton{Model extension}}

\end{frame}


%%%%%%%%%%%%%%%%%%%%%%%%%%%%%%%%%%%%%%%%%%%%%%%%%%%%%%



%%%%%%%%%%%%%%%%%%%%%%%%%%%%%%%%%%%%%%%%%%%%%%%%%%%%%%
%%%%%%%%%%%%%%%%%%%%%%%%%%%%%%%%%%%%%%%%%%%%%%%%%%%%%%



\section{VII. Conclusion}

\begin{frame}{Conclusion}

\begin{itemize}
\item This paper studies the impact of the US monetary policy shocks on China's export prices using exogenous monetary policy shocks and detailed monthly customs data. 
\item A tightening US monetary policy shock significantly increases China's export prices through both the borrowing cost and liquidity channels.
\item Although we focus on the response of exporter behaviors, the finding may also provide enlightenment for other topics such as:
    \begin{itemize}
        \item The global monetary policy spillover on the real economy and asset prices through trade channels;
        \item The impact of international monetary shocks on global inflation in the presence of cross-border trade linkages;
        \item The optimal domestic policy and international coordination in response to adverse global shocks under financial friction.
    \end{itemize}
\end{itemize}

\end{frame}



\begin{frame}{}
  \centering \Huge
   Thank You
\end{frame}


%%%%%%%%%%%%%%%%%%%%%%%%%%%%%%%%%%%%%%%%%%%%%%%%%%%%%%
%%%%%%%%%%%%%%%%%%%%%%%%%%%%%%%%%%%%%%%%%%%%%%%%%%%%%%

\section{Appendix}

\subsection{Extra Tables}

\begin{frame}[label=appendix_tab.altagg]{Alternative aggregation levels of export prices}
\begin{table}[htbp]
    \centering
    \caption{Alternative aggregation levels of export prices}
    \resizebox{\columnwidth}{!}{
    \begin{threeparttable}
    \begin{tabular}{lcccccc}
        \toprule
        & (1)   & (2)   & (3)   & (4)   & (5)   & (6) \\
        & \multicolumn{3}{c}{Firm-product level $\Delta ln P_{hit}$} & \multicolumn{3}{c}{Transaction level $\Delta ln P_{chit}$}  \\
        \midrule
        $brw_t$   & 0.141*** & 0.116*** & 0.122*** & 0.101*** & 0.084*** & 0.086*** \\
              & (0.011) & (0.011) & (0.011) & (0.009) & (0.010) & (0.010) \\
        $\Delta ln P_{chit-1}$&       & 0.275*** & 0.273*** &       & 0.274*** & 0.274*** \\
              &       & (0.002) & (0.002) &       & (0.002) & (0.002) \\
        $Sales_{it-1}$ &       &       & -0.008*** &       &       & -0.007*** \\
              &       &       & (0.001) &       &       & (0.001) \\
        $\Delta ln RER_{ct}$ &       &       &       & 0.072*** & 0.060*** & 0.055*** \\
              &       &       &       & (0.008) & (0.008) & (0.008) \\
        $\Delta ln GDP_{ct}$ &       &       &       & 0.510*** & 0.373*** & 0.404*** \\
              &       &       &       & (0.029) & (0.028) & (0.028) \\
        \midrule
        Firm-product FE & Yes   & Yes   & Yes   & No    & No    & No \\
        Firm-product-country FE & No    & No    & No    & Yes   & Yes   & Yes \\
        N     & 2420021 & 1800404 & 1758304 & 3477990 & 2140293 & 2140293 \\
        \bottomrule
    \end{tabular}
        \begin{tablenotes}
            \footnotesize
            \item Notes: Robust standard errors clustered at the firm level;  *, **, and *** indicate significance at 10\%, 5\%, and 1\% levels. The dependent variables in columns (1)-(3) are changes in firm-product level price, while in columns (4)-(6) are changes in firm-product-country (transaction) level price. All regressions include firm fixed effects.
	\end{tablenotes}
    \end{threeparttable}
    }
    \label{tab.altagg}
\end{table}
\hyperlink{robustness_other}{\beamergotobutton{Back}}
\hyperlink{alt_erpt}{\beamergotobutton{Back:ERPT}}
\end{frame}

\begin{frame}[label=appendix_tab.single]{Alternative sample: only single-product firms}
\begin{table}[htbp]
    \centering
    \caption{Alternative sample: only single-product firms}
    \resizebox{\columnwidth}{!}{
    \begin{threeparttable}
    \begin{tabular}{lcccccc}
        \toprule
        & (1)   & (2)   & (3)   & (4)   & (5)   & (6) \\
        & \multicolumn{3}{c}{Monthly $\Delta ln P_{it}$} & \multicolumn{3}{c}{Annual $\Delta ln P_{it}$}  \\
        \midrule
        $brw_t$   & 0.238*** & 0.191*** & 0.180*** & 0.130*** & 0.183*** & 0.183*** \\
              & (0.021) & (0.022) & (0.024) & (0.020) & (0.033) & (0.033) \\
        $\Delta ln P_{it-1}$ &       & 0.266*** & 0.272*** &       & -0.338*** & -0.339***\\
              &       & (0.005) & (0.006) &       &   (0.018) & (0.018) \\
        $Sales_{it-1}$ &       &       & -0.008** &       &       & 0.005  \\
              &       &       & (0.003) &       &       &  (0.009) \\
        \midrule
        Firm FE & Yes   & Yes   & Yes   & Yes   & Yes   & Yes \\
        N     & 359862 & 231572 & 187476 & 21567 & 8690  & 8690 \\
        \bottomrule
    \end{tabular}
        \begin{tablenotes}
            \footnotesize
            \item Notes: Robust standard errors clustered at the firm level;  *, **, and *** indicate significance at 10\%, 5\%, and 1\% levels. The dependent variables in columns (1)-(3) are changes in monthly price, while columns (4)-(6) are changes in annual price. All regressions include firm fixed effects.
	\end{tablenotes}
    \end{threeparttable}
    }
    \label{tab.single}
\end{table}
\hyperlink{robustness_other}{\beamergotobutton{Back}}
\end{frame}

\begin{frame}[label=appendix_tab.altfe]{Alternative standard error clusters and fixed effects}
\begin{table}[htbp]
    \centering
    \caption{Alternative standard error clusters and fixed effects}
    \resizebox{\columnwidth}{!}{
    \begin{threeparttable}
    \begin{tabular}{lcccccc}
        \toprule
        & (1)   & (2)   & (3)   & (4)   & (5)   & (6) \\
        & \multicolumn{3}{c}{Firm and year clusters} & \multicolumn{3}{c}{Firm and year fixed effects}  \\
        \midrule
        $brw_t$   & 0.181** & 0.145** & 0.147** & 0.042*** & 0.057*** & 0.059*** \\
              & (0.046) & (0.042) & (0.042) & (0.010) & (0.010) & (0.010) \\
        $\Delta ln P_{it-1}$ &       & 0.301*** & 0.299*** &       & 0.299*** & 0.297*** \\
              &       & (0.013) & (0.013) &       & (0.003) & (0.003) \\
        $Sales_{it-1}$ &       &       & -0.004 &       &       & -0.017*** \\
              &       &       & (0.004) &       &       & (0.001) \\
        \midrule
        Firm FE & Yes   & Yes   & Yes   & Yes   & Yes   & Yes \\
        Year FE & No    & No    & No    & Yes   & Yes   & Yes \\
        N     & 1100399 & 940015 & 917435 & 1100399 & 940015 & 917435 \\
        \bottomrule
    \end{tabular}
        \begin{tablenotes}
            \footnotesize
            \item Notes: Robust standard errors clustered at both firm level and year level for columns (1)-(3) while only at the firm level for columns (4)-(6);  *, **, and *** indicate significance at 10\%, 5\%, and 1\% levels. Regressions for columns (1)-(3) include firm fixed effects, while those for columns (4)-(6) include firm fixed effects and year fixed effects.
	\end{tablenotes}
    \end{threeparttable}
    }
    \label{tab.altfe}
\end{table}
\hyperlink{robustness_other}{\beamergotobutton{Back}}
\end{frame}

\begin{frame}[label=appendix_tab.USD]{USD price responses to monetary policy shocks}
\begin{table}[htbp]
    \centering
    \caption{USD price responses to monetary policy shocks}
    \resizebox{\columnwidth}{!}{
    \begin{threeparttable}
    \begin{tabular}{lcccccc}
        \toprule
        & (1)   & (2)   & (3)   & (4)   & (5)   & (6) \\
        & \multicolumn{3}{c}{Monthly $\Delta ln P_{it}$} & \multicolumn{3}{c}{Annual $\Delta ln P_{it}$}  \\
        \midrule
        $brw_t$   & 0.173*** & 0.129*** & 0.127*** & 0.173*** & 0.217*** & 0.216*** \\
              & (0.011) & (0.010) & (0.010) & (0.010) & (0.012) & (0.012) \\
        $\Delta ln P_{it-1}$ &       & 0.301*** & 0.299*** &       & -0.315*** & -0.314*** \\
              &       & (0.003) & (0.003) &       & (0.007) & (0.007) \\
        $Sales_{it-1}$ &       &       & 0.004*** &       &      & -0.008**\\
              &       &       & (0.001) &       &       & (0.004)\\
        Firm FE & Yes   & Yes   & Yes   & Yes   & Yes   & Yes \\
        N     & 1100400 & 940007 & 917428 & 155049 & 97987 & 97987 \\
        \bottomrule
    \end{tabular}
        \begin{tablenotes}
            \footnotesize
            \item Notes: Robust standard errors clustered at the firm level;  *, **, and *** indicate significance at 10\%, 5\%, and 1\% levels. The dependent variables in columns (1)-(3) are changes in monthly price denominated in the US dollar, while columns (4)-(6) are changes in annual price denominated in the US dollar. All regressions include firm fixed effects.
	\end{tablenotes}
    \end{threeparttable}
    }
    \label{tab.USD}
\end{table}
\hyperlink{robustness_other}{\beamergotobutton{Back}}
\end{frame}

\subsection{Model extension}

\begin{frame}[label=appendix_model_extension]{Dynamic optimization and sticky price} 
\hyperlink{model_extension}{\beamergotobutton{Back}}\\
If the borrowing constraint is binding, we can rearrange the constraint and obtain the expression of the export price:

\begin{equation}
p_t=\frac{\sum_{i=0}^{\infty} \lambda^i \Omega_{i,t+i}\frac{Y_{t+i}}{P_{t+i}^{1-\sigma}}\frac{\tau_{t+i}w_{t+i}}{\phi_{t+i}}(R_{t+i}^{-\nu}\zeta_{t+i}+\delta_{t+i})}{\sum_{i=0}^{\infty} \lambda^i \Omega_{i,t+i}\frac{Y_{t+i}}{P_{t+i}^{1-\sigma}}R_{t+i}^{-\nu}}
\end{equation}

If the borrowing constraint is not binding, we solve the unconstrained problem and get the first-order condition:

\begin{equation}
p_t=\frac{\sigma}{\sigma-1}\frac{\mathbb{E}_t \sum_{i=0}^{\infty} \lambda^i \Omega_{i,t+i}\frac{P_t^{-\sigma}}{P_{t+i}^{-\sigma}}Y_{t+i}\varphi_{t+i}}{\mathbb{E}_t \sum_{i=0}^{\infty} \lambda^i \Omega_{i,t+i}\frac{P_t^{-\sigma}}{P_{t+i}^{1-\sigma}}Y_{t+i}}
\end{equation}

\end{frame}

\begin{frame}{Invoicing currency}

\textbf{\textit{DCP}}

If the constraint is binding
$$
pe_{us}=[(1-R^{\nu})\zeta+(1-c^\gamma)R^{\nu}] \frac{\tau w}{\phi}
$$

If the constraint is not binding

$$
pe_{us}=\frac{\sigma}{\sigma-1}\frac{\tau w [c^\gamma+\zeta+(1-c^\gamma-\zeta) R^\alpha]}{\phi}
$$

\textbf{\textit{LCP}}

If the constraint is binding
$$
pe_{j}=[(1-R^{\nu})\zeta+(1-c^\gamma)R^{\nu}] \frac{\tau w}{\phi}
$$

If the constraint is not binding

$$
pe_{j}=\frac{\sigma}{\sigma-1}\frac{\tau w [c^\gamma+\zeta+(1-c^\gamma-\zeta) R^\alpha]}{\phi}
$$

\textbf{\textit{PCP}}
Similar to the benchmark model

\end{frame}



\begin{frame}{Both variable and fixed costs are paid in advance}

If we incorporate fixed costs into the benchmark model and allow a proportion of $\delta$ ($\equiv 1-c^{\gamma}-\zeta$) of both types of costs to be paid in advance. The firm's problem now becomes

$$
\max_{p} \ (p- \frac{\tau w(1-\delta+\delta R^\alpha)}{\phi}) \frac{p^{-\sigma}}{P^{-\sigma}} Y-f
$$

$$
\text{s.t.} \ \theta [(p -\zeta \frac{\tau w}{\phi}) \frac{p^{-\sigma}}{P^{-\sigma}} Y -\zeta f ]\geq(1-c^\gamma-\zeta) (\frac{\tau w}{\phi} \frac{p^{-\sigma}}{P^{-\sigma}} Y+f)
$$

where $f$ is the fixed cost; if the constraint is not binding, the result is similar to the benchmark model result; if the constraint is binding, we can rewrite it as 

\begin{equation}\label{eq:constraint_fixedcost}
p^{1-\sigma}=[(\frac{1-c^{\gamma}-\zeta}{\theta}+\zeta)\frac{\tau w}{\phi}] (p^{1-\sigma})^{\frac{\sigma}{\sigma-1}}+f(\frac{1-c^{\gamma}-\zeta}{\theta}+\zeta)\frac{P^{-\sigma}}{Y}
\end{equation}

\end{frame}


\begin{frame}{Both variable and fixed costs are paid in advance}

When interest rate $R$ increases, liquidity condition $c$ and trade credit $\zeta$ decreases, the curve moves upward, and $p_H^{1-\sigma}$ will be smaller, which yields a higher optimal price.

\begin{figure}[H]
     \centering
         \includegraphics[width=0.8\textwidth]{latex/drafts/pic/fixed_cost.png}
         \caption{\small Both variable and fixed costs are paid in advance}
         \label{fig: fixed_cost}
\end{figure}

\end{frame}

\begin{frame}[allowframebreaks]
\frametitle{References}
    \footnotesize
    \bibliography{latex/setup/reference}
\end{frame}

\end{document}